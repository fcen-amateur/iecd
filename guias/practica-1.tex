\documentclass[12pt]{article}%

\usepackage{amsmath,amsthm, amsfonts, amscd , amssymb}

\usepackage{hyperref}

\usepackage{float}
\usepackage{listings}

\usepackage{graphicx}
\usepackage[usenames, dvipsnames]{color}
\usepackage{enumerate}% http://ctan.org/pkg/enumitem
\usepackage{marvosym} %\Biohazard \Radioactivity \Keyboard \Stopsign
\usepackage{manfnt}  %\dbend, \lhdbend,and \reversedvideodbend.     \textdbend, \textlhdbend,and \usepackage{phaistos} %\PHpedestrian
\usepackage{xcolor}

\setcounter{MaxMatrixCols}{30}

\setlength{\textheight}{23cm} \setlength{\textwidth}{17.5cm}
\setlength{\topmargin}{-1cm} \setlength{\oddsidemargin}{0cm}
\hyphenation{}

\newcommand{\convprob}{ \buildrel{p}\over\longrightarrow}

\def \sp {\textquestiondown}
\def \RR {\mathbb{R}}
\def \EE {\mathbb{E}}
\def \PP {\mathbb{P}}


\begin{document}
	\noindent
	\textbf{Introducci\'on a la Estad\'istica y Ciencia de Datos}\\[0.2cm]\textbf{Pr\'actica 1 - Estad\'istica Descriptiva}
	\vspace{-0.3cm}
	\newline
	\rule{18cm}{0.2mm}


\begin{enumerate}


\item \label{pancreas} El archivo \texttt{Debernardi.csv} contiene los datos referentes a un estudio acerca del c\'ancer de p\'ancreas (m\'as informaci\'on en el archivo \textit{Acerca de los datos}, en el Aula Virtual). 
		\begin{enumerate}
			\item Construir una tabla con los valores observados para la variable \textsc{diagnosis} y su frecuencia relativa.
			\item Realizar un gr\'afico de barras usando la tabla del \'item anterior.
			%\item Graficar la funci\'on de distribuci\'on emp\'irica.
			\end{enumerate}

\item El archivo \texttt{datos\_titanic.csv} contiene informaci\'on sobre una muestra seleccionada al azar de las personas, no tripulantes, que viajaban en el barco tristemente c\'elebre \emph{Titanic}, al momento de su hundimiento en el Oc\'eano Atl\'antico (m\'as informaci\'on en el archivo \textit{Acerca de los datos}, en el Aula Virtual). %Las variables brindadas en el archivo mencionado son:
%\begin{itemize}
%\item \textsc{survived} es dicot\'omica, tomando los valores $\{0,1\}$, el valor 0 indica que la persona no sobrevivi\'o al hundimiento y el 1 indica que se salv\'o. 
%\item \textsc{sex} indica el sexo de la persona.
%\end{itemize}
\begin{enumerate}
\item Estimar la probabilidad de ser mujer sabiendo que sobrevivi\'o y comparar con la estimaci\'on de ser mujer a bordo del \emph{Titanic}.
\item Hacer una tabla de contingencia entre las variables categ\'oricas \textsc{Survived} y \textsc{Pclass}. A partir de esta tabla estimar la probabilidad de sobrevivir dada la clase para los distintos valores de la variable \textsc{Pclass}.
\item Realizar un gr\'afico de barras que vincule a las variables categ\'oricas \textsc{Survived} y \textsc{Pclass}.
\end{enumerate}

\item En un experimento se midi\'o la temperatura de sublimaci\'{o}n del
iridio y del rodio. En los archivos \texttt{iridio.txt} y \texttt{rodio.txt} se encuentran los datos recabados en el experimento.
\begin{enumerate}
\item Comparar los dos conjuntos de datos mediante histogramas y boxplots, graficando los boxplots en paralelo.
\item Hallar las medias, las medianas y las medias podadas al 10\% y 20\% muestrales. Comparar.
\item Hallar los desv\'ios est\'andares, las distancias intercuartiles y las MAD muestrales
como medidas de dispersi\'on.
\item Hallar los cuantiles $0.90$, $0.75$, $0.50$, $0.25$ y $0.10$.
\end{enumerate}

\item En un estudio nutricional se consideran las calor\'ias y el
contenido de sodio de tres tipos de salchichas y se obtuvieron los datos que
se encuentran en los archivos \texttt{salchichas\_A.txt}, \texttt{salchichas\_B.txt} y \texttt{salchichas\_C.txt}.
\begin{enumerate}
\item Armar un archivo que se llame \texttt{salchihas.txt} que contenga toda la información registrada en los tres archivos mencionados agregando una columna que indique el tipo de salchicha en cada caso.	
\item Realizar un histograma para las calor\'ias de cada tipo de salchichas.
\sp Observa grupos en alg\'un gr\'afico?
\sp Cu\'antos grupos observa? \sp Observa alg\'un candidato a dato at\'ipico? \sp Alguno de los histogramas tiene una caracter\'istica particular? %bimodal
\item Realizar los boxplots paralelos para las calor\'ias.
\sp Observa la misma cantidad de grupos que antes?
\sp A cu\'al conclusi\'on llega? De acuerdo con los boxplots graficados, \sp c\'omo caracterizar\'ia la diferencia entre los tres tipos de salchichas desde el punto de vista de las calor\'ias?
\item Repetir con la cantidad de sodio.

%\item \sp Le parece que los datos sustentan la igualdad del contenido medio cal\'orico de las tres variedades?
\end{enumerate}

\item El conjunto de datos que figura en el archivo \texttt{estudiantes.txt} corresponde a 100 determinaciones repetidas de la concentraci\'{o}n de ion nitrato (en $\mu$g/l), 50 de ellas corresponden a un grupo de estudiantes (Grupo 1) y las restantes 50 a otro grupo (Grupo 2).
\begin{enumerate}
\item Estudiar si la distribuci\'on de los conjuntos de datos para ambos
grupos es normal, realizando los correspondientes histogramas y superponiendo
la curva normal. Adem\'as dibujar los qqplots para cada conjunto de datos superponiendo, en otro color, la recta mediante el comando \textsf{qqline}.
\item \sp Le parece a partir de estos datos que ambos grupos
est\'an midiendo lo mismo? Responder comparando medidas de centralidad y de
dispersi\'on de los datos. Hacer boxplots paralelos.
\end{enumerate}


\item Con la finalidad de incrementar las lluvias en zonas des\'{e}rticas,
se desarroll\'{o} un m\'{e}todo que consiste en el bombardeo de la nube con
\'{a}tomos. Para evaluar la efectividad del m\'{e}todo se realiz\'{o} el
siguiente experimento:
\begin{itemize}
\item Para cada nube que se pod\'{\i}a bombardear se decidi\'{o} al azar si se
la trataba o no.
\item Las nubes no tratadas fueron denominadas nubes controles.
\end{itemize}
En el archivo \texttt{nubes.txt} se presentan la cantidad de agua ca\'{\i}da de 26 nubes tratadas y 26
nubes controles.
\begin{enumerate}
\item Realizar boxplots paralelos. \sp Le parece que el
m\'etodo produce alg\'un efecto?
\item Analizar la normalidad realizando qqplots e histogramas (de densidad)
para ambos conjuntos de datos y superponiendo la curva normal. 
\item Realizar la transformaci\'on logaritmo natural a los datos (log en R) y repetir \textit{b}) para los datos transformados.
\item Realizar boxplots paralelos habiendo transformado las variables con el logaritmo natural. Observar c\'omo se modificaron los datos at\'ipicos respecto del \'item \textit{a}).
\end{enumerate}


\item  \textcolor{red}{El archivo \texttt{data\_credit\_card.csv} tiene información de \texttt{n=500} clientes de un banco, para las siguientes variables: \texttt{purchases} es el monto total de compras en el último año, \texttt{credit\_limit} es el límite de crédito disponible para el cliente, \texttt{purchases\_freq} es la proporción de semanas del año en las que el cliente realizó compras y \texttt{tenure} es la cantidad de meses que restan al cliente para cancelar el crédito. Se pide:
\begin{enumerate}
\item  Para todas las variables, graficar la función de distribución empírica. Discutir sobre el tipo de variable aleatoria que utilizaría para modelar en cada caso.
\item Para la variable \texttt{credit\_limit} hacer un histograma y un gráfico de densidad usando la funcion \texttt{density}, ¿Qué observa? ¿Le parece adecuado realizar estos gráficos para las variables \texttt{purchases} y \texttt{tenure}?
\item Para la variable \texttt{tenure} hacer un \textit{barplot} con las frecuencias relativas de cada valor. ¿Qué observa?
\item Para todas las variables, calcular la media, la mediana y la media $\alpha-$podada (con $\alpha=$0.1). Comparar los resultados y justificar. ¿Qué medida de posición del centro de los datos le parece más adecuada en cada caso?
\item Para todas las variables, obtener los cuantiles de nivel 0.25 y 0.75 de los datos. Calcular el rango inter-cuartílico y la MAD muestrales. Graficar \textit{boxplots}. ¿Qué observa?
\item Calcular el desvío estándar, el coeficiente de asimetría y el coeficiente de curtosis muestrales. Interpretar los resultados en relación a las distribuciones vistas.
\item Identificar datos atípicos. ¿Deberían excluirse? ¿Cómo se modifican las medidas obtenidas anteriormente si se los excluye?
\end{enumerate}}

\item \textcolor{red}{ En el archivo \texttt{ciclocombinado.xlsx} hay datos de la potencia entregada por una central térmica de ciclo combinado. Se registraron datos diarios de la potencia máxima entregada (PE, en MW) por la planta funcionando en capacidad máxima. La variable \texttt{HighTemp} vale 1 si la temperatura media diaria fue superior a 20°C en el día en el que se tomó el dato y vale 0 en caso contrario.
\begin{enumerate}
    \item Realizar un histograma y un gráfico \texttt{density} con los datos de PE, ¿Qué se observa? 
    \item Clasificar los datos en dos vectores según la variable \texttt{HighTemp} y realizar gráficos \texttt{density} separados. Visualizar simultáneamente los gráficos en la misma escala.  ¿Qué se observa?
    \item Estimar $P(\texttt{PE} < 450 | \texttt{HighTemp} = 0)$ y $P(\texttt{PE} < 300 | \texttt{HighTemp} = 1)$.
    \item Estimar $P(\texttt{PE} < 450 )$.
    \item Estimar la potencia mínima garantizada con probabilidad 0.9 para un cierto día con \texttt{Hightemp = 1}.
    \item Estimar la potencia mínima garantizada con probabilidad 0.9 para un cierto día.
\end{enumerate}}

\item Considerar nuevamente el conjunto de datos del ejercicio \ref{pancreas}. 
		\begin{enumerate}
			\item Realizar histogramas para la variable \textsc{LYVE1} basados en los datos brindados para las observaciones que cumplen \textsc{diagnosis}=1, \textsc{diagnosis}=2 y \textsc{diagnosis}=3. Es decir efectuar histogramas seg\'un los niveles de la variable factor \textsc{diagnosis}. Indicar las caracter\'isticas m\'as sobresalientes de los histogramas y aquellas que los diferencian.
			\item Graficar, en distintos colores y superpuestas, las funciones de distribuci\'on emp\'iricas de la variable \textsc{LYVE1} seg\'un los niveles de la variable factor \textsc{diagnosis}. Decidir si la siguiente afirmaci\'on es verdadera o falsa y justificar: ``los valores de la variable \textsc{LYVE1} tienden a ser m\'as altos entre quienes tienen c\'ancer de p\'ancreas que entre quienes sufren otras enfermedades asociadas al p\'ancreas".
			\item Realizar boxplots paralelos para la variable \textsc{LYVE1} seg\'un los niveles de la variable factor \textsc{diagnosis}, considerando el sexo de los pacientes (variable \textsc{sex}). Decidir si la siguiente afirmaci\'on es verdadera o falsa y justificar: ``en t\'erminos generales, el sexo del paciente no afecta los niveles de la prote\'ina que se mide en la variable \textsc{LYVE1}".
			\item Graficar superpuestas las densidades estimadas, que brinda la funci\'on \textsf{density}, para la variable \textsc{LYVE1} seg\'un los niveles de la variable factor \textsc{diagnosis}. Describir las caracter\'isticas m\'as sobresalientes de las densidades estimadas y aquellas que las diferencian.
			\item Repetir a) y d) para el logaritmo de \textsc{LYVE1}.
	\end{enumerate}
	

\end{enumerate}


\end{document}

